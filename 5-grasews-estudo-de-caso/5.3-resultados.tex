\section{Resultados}\label{5-estudo-de-caso-resultados}

Neste capítulo demonstramos o uso da ferramenta Grasews para a anotação semântica de uma especificação WSDL. O processo foi inteiramente conduzido por meio das notações visuais providas por Grasews juntamente com a edição colaborativa, envolvendo um desenvolvedor de serviços web e um especialista de domínio.

A utilização da ferramenta facilitou a anotação semântica da especificação WSDL utilizada nesta prova de conceito. Como a especificação WSDL foi criada para o propósito deste estudo de caso, não pudemos comparar os resultados obtidos com outro cenário, onde a especificação já estivesse anotada. O foco desta prova de conceito não foi validar os conceitos ontológicos utilizados na anotação semântica, mas sim validar o processo de anotação semântica auxiliado por notações visuais e pelo trabalho colaborativo por meio da ferramenta Grasews.

Nenhum URI de transformação para os atributos \textit{Lifting Schema Mapping} e \textit{Lowering Schema Mapping} foi utilizado durante o processo de prova de conceito.

A \tablename~\ref{tab:estudo-de-caso-elementos-wsdl-anotados} lista os elementos WSDL/XSD anotados durante a prova de conceito juntamente com os URIs da ontologia \texttt{MovieOntology} utilizados no atributo \textit{Model Reference}. Note que a ontologia \texttt{MovieOntology} utiliza conceitos provindos da ontologia \texttt{dbpedia}.

\begin{table}[ht!]
    \setlength{\tabcolsep}{10pt} % Default value: 6pt
    \renewcommand{\arraystretch}{1.5} % Default value: 1
    \centering
	\caption[Elementos WSDL/XSD anotados durante a prova de conceito.]{\textbf{Elementos WSDL/XSD anotados durante a prova de conceito.}}
	\label{tab:estudo-de-caso-elementos-wsdl-anotados}
	%\resizebox{\textwidth}{!}{
		\begin{tabular}{| >{\columncolor{Gray}}c | >{\centering\arraybackslash} p{9cm} | }
			\hline
            \rowcolor{Gray}
			\textbf{Elemento WSLD/XSD} & \textbf{URI de \textit{Model Reference}}
			\\
			\hline
			{Movie} & {\url{http://www.movieontology.org/2009/11/09/Movie}}
			\\
            \hline
            {Director} & {\url{http://dbpedia.org/page/Film\_Director}}
			\\ 
            \hline
			{Country} & {\url{http://dbpedia.org/ontology/Country}}
			\\
			\hline
			{Producer} & {\url{http://www.movieontology.org/2009/10/01/movieontology.owl\#Producer}}
			\\
			\hline
			{Actor} & {\url{http://dbpedia.org/ontology/Actor}}
			\\
			\hline
			{Award} & {\url{http://www.movieontology.org/2009/10/01/movieontology.owl\#Award}}
			\\
            \hline
		\end{tabular}
	%}
\end{table}