\begin{resumo}[Abstract]
\begin{otherlanguage*}{english}
    
    Web services have become increasingly important for software development. In order to facilitate the search,  composition and reuse of web services, their descriptions can be semantically annotated using definitions provided by an ontology, thus creating the so-called semantic web services. A semantic web service is developed according to different approaches and standards recommended by W3C, such as OWL-S, SAWSDL, and WSMO-Lite. A number of tools are available to support the development of semantic annotations, such as Radiant, Iridescent, EasyWSDL, and OWL-S Editor. However, these tools support the annotation process at a low abstraction level, therefore requiring from users an extensive technical knowledge on XML/WSDL, among other technologies. The semantic annotation task could be facilitated if the semantic annotation process were carried at a higher abstraction level using graphical notations. Additionally, the semantic annotation could also benefit from a collaborative approach. In other words, different people from different backgrounds could individually contribute with the semantic web services creation, regardless of their geographic locations. In that sense, this project aims at investigating the development of semantic web services via graphical notations and in a collaborative way, following the SAWSDL approach.

   \vspace{\onelineskip}
 
   \noindent \textbf{Keywords}: Semantic web service; Ontology; Web service description; Service oriented architecture; Tool for semantic annotation; Grasews; Graphical Annotation for Semantic Web Services; Usability; User experience; Distribution; Collaborative editing.
 \end{otherlanguage*}
\end{resumo}