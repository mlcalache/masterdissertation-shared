\setlength{\absparsep}{18pt} % ajusta o espaçamento dos parágrafos do resumo
\begin{resumo}
  Os serviços web têm se tornado um paradigma cada vez mais importante no desenvolvimento de software. De modo a facilitar a busca, a composição e o reuso de serviços web, descrições de serviços web podem ser anotadas semanticamente com definições de uma ontologia, criando os chamados serviços web semânticos. Um serviço web semântico é desenvolvido segundo diferentes abordagens e padrões recomendados pela W3C, tais como OWL-S, SAWSDL e WSMO-Lite. Diferentes ferramentas que tem por objetivo facilitar a anotação semântica estão disponíveis no mercado, como Radiant, Iridescent, EasyWSDL e OWL-S Editor. Porém, estas ferramentas possuem baixo nível de abstração, exigindo de seus usuários um extenso conhecimento técnico de XML/WSDL e outras tecnologias. Tal tarefa poderia ser facilitada caso o processo de anotação semântica pudesse ser realizado por meio de notações gráficas em um nível maior de abstração. Adicionalmente, a anotação semântica poderia ser beneficiada se feita de forma colaborativa. Em outras palavras, diferentes pessoas com diferentes especializações poderiam contribuir individualmente na criação de serviços web semânticos, independentemente de suas localizações geográficas. Neste sentido, o objetivo deste trabalho é a investigação do desenvolvimento de serviços web semânticos por meio de notações gráficas e de forma colaborativa, segundo a abordagem SAWSDL.

 \textbf{Palavras-chave}: Serviço web semântico; Ontologia; Descrição de serviços web; Arquitetura orientada a serviços; Ferramenta de anotação semântica; Grasews; Graphical Annotation for Semantic Web Services; Usabilidade; Experiência de usuário; Distribuição; Edição colaborativa.
\end{resumo}