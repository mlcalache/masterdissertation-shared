\chapter{Artefatos da Prova de Conceito}\label{anexo-artefatos-estudo-de-caso}

\section{Especificação WSDL original}\label{anexo-WSDL-estudo-de-caso}

A \lstlistingname~\ref{lst:estudo-de-caso-wsdl} apresenta o código da especificação WSDL original utilizado pelo estudo de caso no Capítulo \ref{5-estudo-de-caso} deste trabalho. Note que não há anotação semântica alguma e nem a existência do \textit{namespace} de SAWSDL.

\lstinputlisting[language=xml,caption={[Especificação WSDL do estudo de caso]\textbf{Especificação WSDL do estudo de caso}.},label={lst:estudo-de-caso-wsdl}]{9-pos-textuais/anexos/arquivos/movie-original.wsdl}

%\section{Ontologia OWL}\label{anexo-OWL-estudo-de-caso}

%A \lstlistingname~\ref{lst:estudo-de-caso-wsdl-anotado} apresenta o código da especificação WSDL após a anotação semântica realizada pelo estudo de caso no Capítulo \ref{5-estudo-de-caso} deste trabalho.

%\lstinputlisting[language=xml,caption={[Ontologia OWL do estudo de caso]\textbf{Ontologia OWL do estudo de caso}.},label={lst:estudo-de-caso-owl}]{pos-textuais/anexos/arquivos/movieontology.owl}

\section{Especificação WSDL com Anotação Semântica}\label{anexo-WSDL-anotado-estudo-de-caso}

A \lstlistingname~\ref{lst:estudo-de-caso-wsdl-anotado} apresenta o código da especificação WSDL após a anotação semântica realizada pelo estudo de caso no Capítulo \ref{5-estudo-de-caso} deste trabalho.

\lstinputlisting[language=xml,caption={[Especificação WSDL anotada após o estudo de caso]\textbf{Especificação WSDL anotada após o estudo de caso}.},label={lst:estudo-de-caso-wsdl-anotado}]{9-pos-textuais/anexos/arquivos/movie-anotado.wsdl}