\subsection{Representação do Conhecimento}\label{2-fundamentacao-sws-representacao-conhecimento}

O conhecimento de um dado domínio pode ser formalmente representado por meio de uma conceitualização~\cite{GRUBER-1993-Ontologies, GRUBER-1995-Ontologies}. Uma conceitualização consiste em definições de entidades (objetos) que presume-se existir em uma determinada área de conhecimento (interesse), incluindo as relações existentes entre estes objetos. Assim, uma conceitualização representa uma visão simplificada e abstrata de um domínio de conhecimento. Toda base de conhecimento, todo sistema baseado em conhecimento ou agente baseado em conhecimento estão relacionados à alguma conceitualização, de forma explícita ou implícita.

Uma ontologia representa uma especificação explícita de uma conceitualização. Na Filosofia, uma ontologia representa um relato sistemático da existência de objetos de um dado domínio. Quando o conhecimento de um domínio é representado por meio de um formalismo declarativo, o conjunto formado pelos conceitos que podem ser representados é chamado de universo de discurso~\cite{GRUBER-1993-Ontologies}.

%Este conjunto de objetos e relacionamentos são especificados para representar este universo de discurso.

Na área da computação, uma ontologia é um artefato criado para representar de forma explícita o conhecimento de um domínio por meio de definições de conceitos e relacionamentos entre os mesmos. Uma ontologia é geralmente desenvolvida por especialistas do domínio de interesse. Atividades como análises conceituais e modelagem de domínios com base em metodologias-padrões têm sido elaboradas ao longo dos anos com o objetivo de aprimorar e padronizar o desenvolvimento de novas ontologias~\cite{GUARINO-1998-Ontology}. O estudo de ontologias tem sua importância reconhecida nas áreas de Inteligência Artificial, Linguística Computacional, Teoria de Banco de Dados, entre outras.

Atualmente, um volume crescente de dados tem sido disponibilizado na Web. A Web Semântica tem por objetivo associar dados (documentos) disponíveis na Web a significados bem definidos~\cite{BERNERS-HENDLER-LASSILA-2001-Semantic-Web}. Esta associação facilita a recuperação de informações, a integração e o reuso destes dados, tanto por máquinas quanto por seres humanos~\cite{CARDOSO-2006-Semantic-Web-Services}. Para criar esta realidade, anotações semânticas devem ser adicionadas aos dados contidos em documentos HTML e XHTML da Web. As anotações semânticas são feitas a partir de referências a conceitos definidos em uma ontologia. 

Dentre as linguagens utilizadas para a representação de conhecimento, destaca-se a linguagem \textit{Web Ontology Language} (OWL), atualmente na versão 2.0~\cite{W3C-2012-OWL, W3C-2012-OWL-Primer}. OWL é uma linguagem para a construção de ontologias em diferentes domínios de conhecimento. O conhecimento em uma ontologia OWL é representado por meio da especificação de classes, propriedades, instâncias e valores. Uma ontologia OWL pode ser representada usando-se diferentes sintaxes, tais como \textit{Functional-Style}, RDF/XML, \textit{Manchester}, \textit{Turtle} e OWL/XML. A linguagem OWL faz parte de um conjunto de recursos da Web Semântica, que incluem RDF~\cite{W3C-2014-RDF}, RDFS~\cite{W3C-2007-RDFS}, SPARQL~\cite{W3C-2008-SPARQL}, entre outros.