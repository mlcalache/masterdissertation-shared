\subsection{Abordagem SAWSDL}\label{2-fundamentacao-sws-sawsdl}
%\subsubsection{SAWSDL}\label{2-fundamentacao-sws-abordagens-sawsdl}

\textit{Semantic Annotations for WSDL and XML Schema} (SAWSDL) é um formato padrão da W3C para anotações semânticas em documentos WSDL~\cite{W3C-2007-SAWSDL}. SAWSDL define mecanismos para permitir que anotações semânticas possam ser adicionadas em um documento WSDL a partir de conceitos contidos em uma ontologia. Assim, SAWSDL pode ser visto como um conector, associando descrições puramente sintáticas, representadas por serviços web e suas descrições, a conceitos semânticos, representados por ontologias~\cite{KOPECKY-VITVAR-BOURNEZ-FARREL-2007-SAWSDL}. A partir desta associação, aplicações computacionais podem interpretar (parcialmente ou completamente) esses conceitos, a fim de automatizar tarefas, tais como o descobrimento, a seleção e a composição de serviços web.

SAWSDL é independente de linguagens específicas para a representação de ontologias, bastando que os conceitos semânticos possam ser identificados a partir de URIs. Normalmente, os \textit{frameworks} de anotação semântica para serviços web utilizam OWL/RDF (\textit{Resource Description Framework}) em conjunto com SAWSDL. SAWSDL contém dois tipos de atributos: \textit{Model Reference} e \textit{Schema Mapping}. Ambos são apresentados a seguir.

Um atributo \textit{Model Reference} (\textit{sawsdl:modelReference}) representa uma anotação de um elemento (entidade) de um documento WSDL com referências a um conjunto de conceitos definidos em uma ou mais ontologias~\cite{KOPECKY-VITVAR-BOURNEZ-FARREL-2007-SAWSDL}. O valor contido no atributo \textit{Model Reference} é um conjunto de zero ou mais URIs, separados por espaços, que identificam, cada qual um diferente conceito~\cite{W3C-2007-SAWSDL}. Dada a definição de uma interface WSDL, o atributo \textit{Model Reference} pode ser utilizado para prover uma descrição semântica para a interface (\textit{<wsdl:interface>}) como um todo e/ou para cada operação (\textit{<wsdl:operation>}) definida nesta interface.

Embora os nomes dos elementos de entrada e de saída possam indicar o comportamento esperado de uma operação, tal objetivo nem sempre pode ser atingido dada a possível incerteza e ambiguidade intrínseca aos identificadores utilizados. Neste sentido, cada elemento de uma entrada, saída ou falha associado a cada operação definida pode ser individualmente anotado com o atributo modelReference para prover uma descrição semântica para este elemento. A anotação de uma falha é composta por uma referência a um conceito semântico que provê uma descrição geral (alto nível) da falha. A anotação da ocorrência de uma falha não descreve a mensagem da falha em si. Esta mensagem deve ser criada como um elemento XSD, do documento \textit{XML Schema}, e, portanto, este elemento também pode ser anotado. 

O atributo \textit{Model Reference} também pode ser utilizado para anotar os elementos pertencentes ao documento \textit{XML Schema} (XSD), como, por exemplo, \textit{<xs:element>}, \textit{<xs:attribute>}, \textit{<xs:simpleType>} e \textit{<xs:complexType>}. Os elementos \textit{<xs:element>} e \textit{<xs:attribute>} podem ser anotados com \textit{Model Reference}, com o propósito de descrever um conceito definido em um modelo semântico. Ao anotar um elemento \textit{<xs:simpleType>}, qualquer elemento ou atributo deste tipo simples também será associado ao conceito do modelo semântico referenciado.

Um elemento \textit{<xs:complexType>} pode ser anotado segundo duas abordagens distintas: \textit{Bottom Level Annotation} e \textit{Top Level Annotation}. \textit{Bottom Level Annotation} refere-se à anotação aplicada a um elemento, que compõe o tipo complexo, ou a um atributo. Deste modo, todos os elementos e atributos de um tipo complexo podem ser anotados, caracterizando, portanto, uma anotação mais granularizada e específica para os elementos-filhos. Já \textit{Top Level Annotation} refere se à anotação do elemento complexo como um todo, caracterizando, portanto, uma anotação mais genérica, sem se especializar nos elementos e atributos contidos no compartimento-pai. Apesar da existência de duas abordagens, elas são independentes, podendo ser utilizadas simultaneamente em um elemento \textit{<xs:complexType>} (\textit{Top Level Annotation}) e seus elementos-filhos (\textit{Bottom Level Annotation}).

Um atributo do tipo \textit{Schema Mapping} é utilizado para tratar eventuais incompatibilidades existentes entre o modelo semântico (ontologia) e a estrutura de parâmetros de entrada (\textit{input}) e saída (\textit{output}) de um serviço web (WSDL). O atributo \textit{Schema Mapping} permite que transformações (conversões) de tipos de dados contidos em um serviço web e em uma ontologia sejam realizadas por meio de métodos \textit{Lifting} e \textit{Lowering}.

O método \textit{Lifting}, representado pelo atributo \textit{sawsdl:liftingSchemaMapping}, permite que dados presentes em um nível sintático (WSDL/XML) sejam mapeados para um nível semântico. Neste sentido, \textit{Lifting} permite que dados no formato XML, produzidos por um serviço web, sejam transformados em instâncias de um modelo semântico, como, por exemplo, instâncias de classes OWL. O método \textit{Lowering}, representado pelo atributo \textit{sawsdl:loweringSchemaMapping}, permite que dados possam ser manipulados de forma ontológica, i.e., \textit{Lowering} transforma os tipos de dados de um nível semântico em tipos de dados de um nível sintático (WSDL/XML). Neste sentido, \textit{Lowering} permite que instâncias de classes OWL possam ser transformadas de volta em dados no formato XML.

Cabe ressaltar que o SAWSDL não especifica a forma como a transformação dos dados sintáticos para os semânticos (ou vice-versa) será realizada. Os atributos \textit{sawsdl:liftingSchemaMapping} e \textit{sawsdl:loweringSchemaMapping} apenas fazem referência a um URI contendo um conjunto de regras de transformação. Estas regras podem ser definidas usando, por exemplo, as linguagens \textit{EXtensible Stylesheet Language Transformation} (XSLT)~\cite{W3C-1999-XSLT} e XQuery~\cite{W3C-2017-XQuery}.