\subsection{Definição de Serviços Web Semânticos}\label{2-fundamentacao-sws-definicao}

O Desenvolvimento Baseado em Serviços pressupõe o reuso e a integração de serviços (composição de serviços). Porém, como identificar se um serviço atende às necessidades (funcionais) de um dado cliente e como identificar e resolver eventuais problemas de incompatibilidade semântica de modo a obter a interoperabilidade desejada entre diferentes serviços? Para responder a tais questionamentos, são necessárias informações precisas sobre as funcionalidades de um determinado serviço, bem como sobre os dados de entrada e de saída deste serviço.

Serviços web semânticos (SWS) representam a extensão de serviços web com a aplicação dos princípios da Web Semântica. SWS representam serviços web cujas interfaces são semanticamente anotadas com conceitos (termos) definidos em uma ontologia. Neste sentido, SWS são criados a partir da adição de definições semânticas a interfaces de serviços web, bem como a operações e a dados de entrada e saída destas operações~\cite{CARDOSO-2006-Semantic-Web-Services, W3C-2007-SAWSDL, STAVROPOULOS-2013-Iridescent}.

O desenvolvimento de SWS proporciona diferentes benefícios, dentre os quais destacam-se~\cite{SHETH-2007-SAWSDL-Tools}: i) facilidade de reuso de serviços, dado que descrições semânticas colaboram para encontrar serviços mais relevantes; ii) interoperabilidade semântica, alcançada por meio da anotação e do mapeamento semântico realizado entre os dados trocados entre serviços web; iii) facilidade de configuração e composição, permitindo ligações dinâmicas entre diferentes serviços web; iv) e, por fim, nível mais elevado de automação no ciclo de vida de uma atividade desenvolvida por uma organização, ou seja, auxilia na configuração dinâmica (análises de descoberta e restrições) e execução (tratamento de exceções durante a execução) dos processos desenvolvidos por uma organização.