\subsection{Desenvolvimento Colaborativo de Software}\label{2-fundamentacao-sistemas-colaborativos-desenvolvimento}

Sistemas computacionais tornaram-se vitais para a sociedade contemporânea~\cite{SOUZA-MARCZAK-PRIKLANDNICKI-2012-Desenvolvimento-Colaborativo-Software}. Diferentes organizações e negócios estão cada vez mais dependentes de funcionalidades fornecidas por estes sistemas, os quais têm se tornado cada vez mais complexos. A fim de atender aos desafios impostos por este aumento de complexidade, equipe de profissionais normalmente trabalham em conjunto no desenvolvimento de sistemas computacionais complexos. Estes profissionais trabalham colaborativamente para produzir soluções com sucesso, i.e., com qualidade, eficiência e eficácia. Entre estes especialistas, alguns papéis podem ser citados, como desenvolvedores, analistas de requisitos, arquitetos de \textit{softwares}, analistas de negócio, gerentes de projetos, entre outros. Os membros de uma equipe de desenvolvimento de \textit{software} precisam coordenar suas atividades, planejar novas ações, tomar decisões, realizar as atividades previstas e, também, comunicar-se com os demais membros da equipe.

O desenvolvimento de um sistema computacional requer a criação (colaborativa) de diferentes tipos de artefatos, tais como a especificação (textual) de requisitos do sistema, diagramas de casos de uso e de classe UML, e o código-fonte do sistema em uma dada linguagem de programação. Mudanças em quaisquer destes artefatos devem ser alinhadas (sincronizadas) corretamente de modo a evitar erros na interpretação, construção e execução deste sistema computacional.

Diversas ferramentas de desenvolvimento de \textit{software} dão suporte à colaboração entre os profissionais envolvidos no processo de desenvolvimento. Estas ferramentas dão suporte desde a especificação colaborativa de requisitos até a construção de diagramas e códigos-fonte. O suporte pode ocorrer de maneira síncrona ou assíncrona, passando por editores colaborativos de texto, que possibilitam que diferentes desenvolvedores de \textit{software} escrevam o código-fonte ao mesmo tempo, até sistemas que possibilitam gerenciar o acesso compartilhado de artefatos computacionais entre um determinado grupo de usuários (profissionais).

Durante o processo de edição colaborativa de um documento qualquer, a comunicação é essencial, visto que os autores devem acordar com o que será feito e também reportar (informar) o que foi feito após a conclusão de cada parte escrita (tarefa concluída). A edição colaborativa é regularmente aplicada em documentos textuais ou documentos de código-fonte de \textit{softwares}. Contribuições assíncronas e remotas são muito eficientes, visto que os atores não necessitam se reunir a fim de realizar o trabalho (edição) colaborativamente. Porém, o gerenciamento das tarefas deve ser realizado com muito cuidado e atenção, envolvendo a divisão e a distribuição das tarefas entre os colaboradores envolvidos no processo. As tarefas podem ser feitas sequencialmente, a fim de que não haja problemas de sincronização entre os colaboradores, ou estas podem ser realizadas simultaneamente (forma síncrona). De qualquer maneira, o processo de edição deve ser minuciosamente planejado, documentado e revisado.

O desenvolvimento de um serviço web semântico segundo a abordagem SAWSDL tem como pressuposto a criação de anotações semânticas em um documento WSDL. A edição deste documento pode também ser realizada de forma colaborativa, envolvendo não apenas profissionais da área de tecnologia da informação (TI), mas também especialistas de domínio. A participação de um especialista de domínio pode ser fundamental no processo de anotação semântica, haja visto seu conhecimento sobre um determinado domínio de conhecimento. Este especialista pode eventualmente não ser um profissional da área de TI.



%~\cite{SOUZA-MARCZAK-PRIKLANDNICKI-2012-Desenvolvimento-Colaborativo-Software}
%O TCSC também pode ser utilizado para o desenvolvimento de sistemas, visto que modelos e códigos-fonte de \textit{software} são geralmente escritos por desenvolvedores de \textit{software} da mesma forma que documentos de texto são redigidos por seus autores. Modelos textuais e visuais são artefatos que servem como abstrações do \textit{software} que está sendo construído. Exemplos de modelos textuais incluem especificações de requisitos. Exemplos de modelos visuais incluem diagramas, como diagramas UML de casos de uso, de classe e de sequência, diagramas entidade relacionamento, entre outros.

%O objetivo de tais artefatos é dar suporte à construção efetiva do código-fonte do \textit{software} em uma ou mais linguagens de programação. Os modelos de \textit{software} variam de acordo com o grau de formalismo, sendo classificados em: formal, como os sistemas escritos em linguagens de programação; semiformal, como diagramas; ou, por fim, informal, como a linguagem natural utilizada nas especificações de requisitos. Cada modelo é utilizado em uma ou mais fases do processo de desenvolvimento de \textit{software} (engenharia de \textit{software}), como a análise, projeto, a codificação, os testes e a implantação.

%~\cite{SOUZA-MARCZAK-PRIKLANDNICKI-2012-Desenvolvimento-Colaborativo-Software}
%Diferentes papeis da área de tecnologia da informação (TI) podem trabalhar colaborativamente no desenvolvimento de \textit{softwares}. A coordenação de um trabalho colaborativo é de extrema importância no desenvolvimento colaborativo de \textit{softwares}. Mudanças no código-fonte devem ser alinhadas (sincronizadas) corretamente. Caso os modelos não estejam sincronizados, i.e., forem diferentes, erros podem ocorrer na construção (compilação), interpretação ou execução do código-fonte.

%Comumente, a estrutura e o conteúdo do documento tem o envolvimento de todos os autores.

%Collaborative writing is writing done by more than one person; they may discuss what they are going to write before they start, and discuss what they have written after they finish each draft they write.[2] The typing might be organized by dividing the writing into sub-tasks assigned to each group member, with the first part of the tasks done before the next parts, or they might work together on each task.[3][4] The writing is planned, written, and revised, and more than one person is involved in at least one of those steps.[5] Usually, discussions about the document's structure and context involve the entire group.[6]

%A wiki is a knowledge base website on which users collaboratively modify content and structure directly from the web browser. In a typical wiki, text is written using a simplified markup language and often edited with the help of a rich-text editor. A wiki is run using wiki software, otherwise known as a wiki engine
    
%Most usually it is applied to textual documents or programmatic source code. Such asynchronous (non-simultaneous) contributions are very efficient in time, as group members need not assemble in order to work together. Generally, managing such work requires software;[7] the most common tools for editing documents are wikis, and those for programming, version control systems.[8] Most word processors are also capable of recording changes; this allows editors to work on the same document while automatically clearly labeling who contributed what changes. New writing environments such as Google Docs provide collaborative writing/editing functionalities with revision control, synchronous/asynchronous editing. Another tool that uses collaborative editing is Addteq's Excellentable. Excellentable is an app for Confluence that enable users to collaborate on spreadsheets in real-time directly inside of Confluence.
    
%Wikipedia is an example of a collaborative editing project on a large scale, which can be both good and bad, because of the large contributions by the public, Wikipedia has one of the widest ranges of material in the world. Unfortunately, this also leads to online 'graffiti', in which members of the public can submit incorrect information or random rubbish. Collaborative writing can lead to projects that are richer and more complex than those produced by individuals. Many learning communities include one or more collaborative assignments. However, writing with others also makes the writing task more complex.[9] There is increasing amount of research literature investigating how collaborative writing can improve learning experiences.[10]
    
%Correct access management systems can prevent duplicated information.[11] Access management systems require access to a server, often online.[12] Online collaboration can be more difficult due to issues such as time zones.[13]

%\subsubsection{Edição Colaborativa Assíncrona}\label{2-fundamentacao-sistemas-colaborativos-abordagens-assincrona}

%\hl{Estou escrevendo ainda}

%A edição colaborativa assíncrona, também chamada de não-simultânea ou de não-tempo real...

%Uma forma muito utilizada para se implementar a edição colaborativa assíncrona é através do controle de versão~\cite{TILIGADAS-2016-Collaboration}.

%\subsubsection{Edição Colaborativa em Tempo Real}\label{2-fundamentacao-sistemas-colaborativos-abordagens-tempo-real}

%\hl{Estou escrevendo ainda}

%Referencias
%\cite{YANG-SUN-ZHANG-JIA_Realtime-Cooperative-Editing-Internet}
%https://en.wikipedia.org/wiki/Collaborative_real-time_editor}{Collaborative real-time editor from wikipedia

%A primeira aparição do conceito de um editor colaborativo em tempo real foi realizada por Douglas Engelbart no ano de 1968, na apresentação \textit{The Mother of All Demos} ~\cite{ENGELBART-2019-Doug-Great-Demo-1968}. Apesar do conceito ter sido apresentado por Engelbart, as primeiras implementações só foram realizadas décadas depois.

%Com o fenômeno da Web 2.0, surgiu um grande interesse no desenvolvimento de editores colaborativos em tempo real baseados nos navegadores web. 