\subsection{Visão Geral}\label{2-fundamentacao-notacao-visual-visao-geral}

Notações visuais são uma das formas mais antigas e efetivas de representação do conhecimento~\cite{DAVIES-1990-Egyptian-Hieroglyphs, MATHEWS-Classic-1991-Maya_Emblem_Glyphs, KAMESWARA-2005-Prehistoric-Astronomy-India}. Diferentes tipos de recursos visuais, tais como formas geométricas, cores, ícones, texturas e brilhos, podem ser utilizados para representar diferentes conceitos e seus relacionamentos em um dado domínio~\cite{SMITH-MORIARTY-KENNEY-BARBATSIS-2004-Handbook-Visual-Communication, MOODY-2009-Physics-Notation}.

Notações visuais são mais eficazes para a comunicação e a transmissão de informações do que outras formas de comunicação, incluindo a comunicação verbal e textual~\cite{MOODY-2009-Physics-Notation}. Tal característica advém da melhor capacidade do cérebro humano em processar (paralelamente) representações visuais, que utilizam arranjos espaciais de elementos gráficos (bidimensionais). Cerca de um quarto do cérebro humano é dedicado à visão. Assim, o sistema visual de um cérebro humano tem a capacidade de processar mais informações do que todos os demais sentidos combinados.

%Por outro lado, o processamento de notações textuais não é tão eficaz, dado que notações textuais utilizam sequências de caracteres para representações unidimensionais (lineares) e são processadas em série pelo cérebro humano.

%Adicionalmente, o cérebro humano possui sistemas separados com diferentes propósitos, como, por exemplo, o sistema visual para o processamento de representações visuais e o sistema auditivo para o processamento de representações verbais. Cerca de um quarto do cérebro humano é dedicado à visão. O sistema visual de um cérebro humano tem a capacidade de processar mais informações do que todos os demais sentidos combinados.

Atualmente, notações visuais tem sido amplamente utilizadas em diversas áreas de conhecimento. Por exemplo, \textit{Unified Modeling Language} (UML)~\cite{OMG-2017-UML} tem sido utilizada na representação de artefatos de \textit{software}; \textit{Business Process Management Notation} (BMPN)~\cite{OMG-2011-BPMN} tem sido utilizada na representação de processos de negócio; \textit{Systems Biology Graphical Notation} (SBGN)~\cite{NOVERE-BUCKA-MI-MOODIE-SCHREIBER-SOROKIN-2009-SBGN, VASUNDRA-LENOVERE-WALTEMATH-WOLKENHAUER-2018-SBGN} tem sido utilizada na representação de modelos biológicos \textit{in silico}; enquanto que \textit{Synthetic Biology Open Language} (SBOL)~\cite{QUINN-COX-ADLER-2015-SBOL} tem sido utilizada na representação de modelos de engenharia genética.