\definecolor{Gray}{gray}{0.9}

\newcommand{\myhref}[3][red]{\href{#2}{\color{#1}{#3}}}
\newcommand{\mycite}[2][red]{{\hypersetup{citecolor=#1}\cite{#2}}}

% Pacote para gerar textos Lorem Ipsum
\usepackage{lipsum}  

% Pacote para exibição de conteúdos na orientação paisagem
\usepackage{lscape}

% Pacote para uso do highlight \hl command
\usepackage{soul}

% Pacote para usar duas figuras e duas captions juntas
\usepackage{subcaption}

% Pacote para as legendas ficarem em negrito
\usepackage[labelfont=bf]{caption}

% Pacote para blocos coloridos
\usepackage{tcolorbox}

% Pacote para tabs spaces
\usepackage{tabto}

% Pacotes para desenhar grafos gerados pelo http://vowl.visualdataweb.org/webvowl.html
%\usepackage{tikz} 
%\usepackage{helvet} 
%\usetikzlibrary{decorations.markings,decorations.shapes,decorations,arrows,automata,backgrounds,petri,shapes.geometric} 
%\usepackage{xcolor} 

%\usepackage[htt]{hyphenat}

% Pacote para quebrar URLs
\usepackage{xurl}

% If you are using the default LaTeX fonts (or a small number of others), you can enable hyphenation within \texttt using the hyphenat package:
\usepackage[htt]{hyphenat}

\renewcommand{\texttt}[1]{%
  \begingroup
  \ttfamily
  \begingroup\lccode`~=`/\lowercase{\endgroup\def~}{/\discretionary{}{}{}}%
  \begingroup\lccode`~=`[\lowercase{\endgroup\def~}{[\discretionary{}{}{}}%
  \begingroup\lccode`~=`.\lowercase{\endgroup\def~}{.\discretionary{}{}{}}%
  \catcode`/=\active\catcode`[=\active\catcode`.=\active
  \scantokens{#1\noexpand}%
  \endgroup
}

%%%%%%%%%%%%%%%%%%%%%%%%%%%%%%%%%%%%%%%%%%%%%%%%%%%
%% TABELAS %%
%%%%%%%%%%%%%%%%%%%%%%%%%%%%%%%%%%%%%%%%%%%%%%%%%%%
% pacote e comando para criar uma tabela com colunas centralizadas e com tamanho fixo
% o comando 'x' deve ser utilizado
% exemplo: 
%\begin{table}[h]
%    \begin{tabular}{ | x{2cm} | x{5cm} | }
%        \hline  
%        \textbf{a} & \textbf{b}
%        \\ \hline
%    \end{tabular}
%\end{table}
\usepackage{array}
\newcolumntype{x}[1]{>{\centering\arraybackslash\hspace{0pt}}p{#1}}

\usepackage{multirow}
\usepackage{colortbl}
\newcolumntype{g}{>{\columncolor{Gray}}c}
\definecolor{Gray}{gray}{0.9}

%%%%%%%%%%%%%%%%%%%%%%%%%%%%%%%%%%%%%%%%%%%%%%%%%%%%%%%%%%%%%%%%
%% LISTAGENS %%
%%%%%%%%%%%%%%%%%%%%%%%%%%%%%%%%%%%%%%%%%%%%%%%%%%%%%%%%%%%%%%%%

\definecolor{codegreen}{rgb}{0,0.6,0}
\definecolor{codegray}{rgb}{0.5,0.5,0.5}
\definecolor{codepurple}{rgb}{0.58,0,0.82}
\definecolor{backcolour}{rgb}{0.95,0.95,0.92}

\lstdefinestyle{mystyle}{
    backgroundcolor=\color{backcolour},
    commentstyle=\color{codegreen},
    keywordstyle=\color{magenta},
    numberstyle=\tiny\color{codegray},
    stringstyle=\color{codepurple},
    basicstyle=\ttfamily\tiny,
    breakatwhitespace=false,
    breaklines=true,
    captionpos=b,
    keepspaces=true,
    numbers=left,
    numbersep=5pt,
    showspaces=false,
    showstringspaces=false,
    showtabs=false,
    tabsize=2
}

\usepackage{chngcntr}

\renewcommand{\lstlistingname}{Listagem}% Muda o texto de Listing para Listagem

\makeatletter
\AtBeginDocument{
    \renewcommand\lstlistoflistings{
        \bgroup \let\contentsname\lstlistlistingname \def\l@lstlisting##1##2{\@dottedtocline{1}{0em}{3em}{Listagem ##1}{##2}} \let\lst@temp\@starttoc \def\@starttoc##1{\lst@temp{lol}}
        \tableofcontents \egroup} 
} \makeatother
%\renewcommand{\lstlistlistingname}{Lista de \lstlistingname s}% List of Listings -> List of Códigos

\newlistentry{listagem}{lol}{0}

\lstset{style=mystyle}

\lstset{numberbychapter=false}