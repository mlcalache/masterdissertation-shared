Linguagens de modelagem são comumente utilizadas em uma organização para modelar desde regras de negócio até artefatos de \textit{software}. Diferentes princípios propostos pela Física das Notações~\cite{MOODY-2009-Physics-Notation} podem ser utilizados para criar representações visuais de um dado domínio de forma mais eficaz, contribuindo, portanto, para a criação de modelos de mais fácil compreensão.

Este capítulo apresenta uma proposta de notação visual para o desenvolvimento de serviços web semânticos utilizando o padrão SAWSDL. O capítulo está estruturado da seguinte forma: a seção \ref{3-notacao-visual-wsdl} apresenta a notação visual proposta para a representação de elementos de uma especificação WSDL; a seção \ref{3-notacao-visual-owl} apresenta a notação visual proposta para a representação de classes de ontologias OWL; a seção \ref{3-notacao-visual-sawsdl} apresenta a notação visual proposta para anotações semânticas segundo o padrão SAWSDL; e, por fim, a seção \ref{3-adequacao-aos-principios-da-fisica-das-notacoes} avalia a adequação entre as notações visuais propostas e os princípios da Física das Notações.

% o uso de notações visuais que possibilitam abstrair detalhes técnicos (sintáticos) de elementos que compõem uma descrição de serviço web segundo a especificação WSDL~\cite{W3C-2007-WSDL}, de instâncias de classes de uma ontologia OWL~\cite{W3C-2012-OWL}, bem como de anotações semânticas segundo a abordagem SAWSDL~\cite{W3C-2007-SAWSDL}