\section{Adequação aos princípios da Física das Notações}\label{3-adequacao-aos-principios-da-fisica-das-notacoes}

Dentre os nove princípios da Física das Notações, os princípios Clareza Semiótica, Diferenciação Perceptiva, Expressividade Visual, Codificação Dupla e Economia Gráfica foram aplicados ao desenvolvimento da notação visual para SAWSDL. Os demais princípios não possuem aplicação para o propósito deste trabalho.

%\textbf{Princípio da Clareza Semiótica}

O princípio da Clareza Semiótica foi aplicado de modo que elementos de uma especificação WSDL e classes de uma ontologia possuam uma representação visual clara. Todos os elementos de uma especificação WSDL são visualmente representados com o propósito de facilitar o processo de anotação semântica. Os elementos \textit{wsdl:input}, \textit{wsdl:output}, \textit{wsdl:infault} e \textit{wsdl:outfault}, apesar de não serem passíveis de anotação semântica, são visualmente representados de modo a facilitar a compreensão da especificação WSDL. Em relação às classes OWL, classes que não são utilizadas em uma anotação semântica ou que não sejam hierarquicamente dispostas entre classes que são utilizadas em uma anotação semântica não são visualmente representadas.

%\textbf{Princípio da Diferenciação Perceptiva}

O princípio da Diferenciação Perceptiva foi aplicado de modo a tornar as notações visuais propostas por este trabalho claramente distinguíveis umas das outras. Esta distinção ocorre por meio do uso (combinado) dos princípios da Expressividade Visual e da Codificação Dupla, ambos explicados a seguir. Desta forma, entende-se que os diferentes elementos visuais representados no grafo são facilmente identificados pelo usuário.

%\textbf{Princípio da Expressividade Visual}

O princípio da Expressividade Visual foi satisfeito visto que das sete variáveis propostas por este princípio, seis foram utilizadas neste trabalho. A variável posição é aplicada entre os diferentes tipos de elementos WSDL e XSD, de modo a representar quais elementos são compostos por outros elementos da especificação WSDL. A variável tamanho é aplicada entre os diferentes tipos de elementos WSDL conforme seus níveis. A variável brilho foi utilizada a fim de diferenciar elementos WSDL/XSD anotados semanticamente dos elementos não anotados. A variável cor foi utilizada na representação de todos elementos do grafo, a fim de facilitar a diferenciação entre diferentes tipos de elementos da especificação WSDL, de elementos de classes OWL e, por fim, de anotações semânticas (arestas entre elementos WSDL/XSD e classes OWL). A variável orientação foi utilizada nos sentidos das arestas (setas). Entre elementos WSDL/XSD, a orientação da seta diferencia elementos que são compostos por outros elementos. Já entre elementos OWL, a orientação da seta diferencia elementos que especializam outros elementos. Finalmente, o uso da variável forma contribui para a diferenciação entre elementos WSDL (interfaces, operações e mensagens) e elementos XSD (tipos complexos e simples) de uma especificação WSDL e elementos OWL (classes de uma ontologia OWL). Apenas a variável textura não foi utilizada no contexto deste trabalho.

%\textbf{Princípio da Codificação Dupla}

O princípio da Codificação Dupla foi aplicado por meio do uso de diferentes estereótipos juntamente com as representações visuais. Desta forma, a compreensão dos diferentes tipos de elementos visualmente representados no grafo é facilitada.

%\textbf{Princípio da Economia Gráfica}

Finalmente, o princípio da Economia Gráfica foi satisfeito, pois utilizamos uma quantidade mínima de elementos visuais de modo a facilitar a compreensão de uma especificação WSDL e de elementos envolvidos na anotação semântica segundo o padrão SAWSDL. Os tipos elementos que possuem representações visuais variam entre: \textit{wsdl:interface}, \textit{wsdl:operation}, \textit{wsdl:input}, \textit{wsdl:output}, \textit{wsdl:infault}, \textit{wsdl:outfault}, \textit{wsdl:fault}, \textit{xs:complexType} e \textit{xs:simpleType} para elementos WSDL/XSD; e classes para elementos de uma ontologia OWL.

%Os princípios do Ajuste Cognitivo (seção \ref{2-fundamentacao-notacao-visual-principio-ajuste-cognitivo}), da Complexidade Gerenciável (seção \ref{2-fundamentacao-notacao-visual-principio-complexidade-gerenciavel}) e da Transparência Semântica (seção \ref{2-fundamentacao-notacao-visual-principio-transparencia-semantica}) não foram satisfeitos totalmente. O princípio do Ajuste Cognitivo não foi necessário satisfazer pois o público alvo é bem exclusivo: ou desenvolvedores de serviços ou especialistas de domínio. O princípio da Complexidade Gerenciável não foi necessário satisfazer pois o diagrama já possui uma grande Economia Gráfica e a pequena quantidade elementos passíveis de serem representados para a anotação semântica já limitou, consequentemente, a quantidade de recursos visuais necessários na representação visual. Portanto, não houve necessidade de aplicarmos a modularização nem a hierarquização do modelo. Por fim, o princípio Transparência Semântica não foi possível ser satisfeito dada a limitação de customização dos elementos visuais disponíveis biblioteca de desenvolvimento utilizada neste trabalho. Neste sentido, não chegamos a um consenso de customizações visuais necessárias a fim de atender a este princípio e que, ao mesmo tempo, possuíssem uma interface gráfica com boa usabilidade.