\section{Motivação}\label{1-introducao-motivacao}

%a anotação semântica poderia ser realizada em um nível maior de abstração, omitindo dados sintáticos de tecnologias e padrões envolvidos no desenvolvimento de serviços web semânticos. Adicionalmente, a anotação semântica de serviços web poderia ser realizada de forma colaborativa, envolvendo desenvolvedores de serviços e especialistas de domínio. Desta forma, cada indivíduo poderia contribuir na anotação semântica conforme a sua área de conhecimento, tornando o desenvolvimento de serviços web semânticos mais eficiente e eficaz.  

%Contudo, atualmente, o desenvolvimento de serviços web semânticos possui uma grande carência de suporte ferramental. As ferramentas disponíveis possuem baixo nível de abstração, exigindo um alto nível de conhecimento técnico de seus usuários, em relação às tecnologias e aos padrões envolvidos. A utilização de recursos e notações gráficas poderia facilitar o uso de tais ferramentas e, consequentemente, a criação de serviços web semânticos. Adicionalmente, a distribuição destas ferramentas é considerada um grande desafio, visto que elas são disponibilizadas como ferramentas \textit{standalone}, \textit{plugins} ou bibliotecas de desenvolvimento. Tal processo seria facilitado se as ferramentas fossem disponibilizadas como aplicações web. Por fim, nenhuma ferramenta possui suporte à anotação semântica de forma colaborativa, ou seja, nenhuma ferramenta permite que diferentes pessoas com diferentes especializações possam contribuir simultaneamente com a anotação semântica a partir de diferentes geolocalizações.

%A simplicidade encontrada na Internet e os padrões abertos adotados para o desenvolvimento de aplicações na Web tornaram este ambiente bastante popular~\cite{PAPAZOGLOU-GEORGAKOPOULOS_Service-Oriented-Computing}.

%%%%%%%%%%%%%%%%%%%%%%%%%%%%%%%%%%%%%%%%%%%%%%%%%%%
% Contextualização
%%%%%%%%%%%%%%%%%%%%%%%%%%%%%%%%%%%%%%%%%%%%%%%%%%%

Nos últimos anos, presenciamos o surgimento de novos modelos arquitetônicos para o desenvolvimento de aplicações na Web~\cite{PAPAZOGLOU-GEORGAKOPOULOS-2003-Service-Oriented-Computing}. Em particular, podemos destacar o Desenvolvimento Baseado em Serviços (DBS)~\cite{IEEE-2011-ARQUITETURA, SOA-MANIFESTO-2013, IBM-2019-SOA}. De acordo com este modelo arquitetônico, aplicações podem ser desenvolvidas a partir da composição de um conjunto de serviços desenvolvidos por diferentes organizações. Serviços web representam uma solução tecnológica baseada em padrões abertos da Web amplamente utilizada em diferentes domínios do conhecimento para o desenvolvimento baseado em serviços~\cite{DUTTA-2017-Online-Census,GUARDIA-FARIAS-2017-SemantiSCo,KACI-2018-Mobile-Road-Safety,OYUCU-POLATI-2018-Online-Video-Content-Analysis-System,HUANG-JAYARAMAN-MORSHED-2019-Sens-E-Montion}. Serviços web são descritos por meio de descrições de serviços, normalmente desenvolvidas usando o padrão \textit{Web Service Description Language} (WSDL)~\cite{W3C-2007-WSDL}.

De modo a facilitar a descoberta, o reuso e a composição de serviços web, tanto por computadores quanto por seres humanos, descrições de serviços web podem ser anotadas semanticamente por meio de associações a conceitos de uma ontologia. Uma ontologia é um artefato computacional criado para representar de forma explícita o conhecimento de um domínio por meio de definições de conceitos e relacionamentos entre os mesmos~\cite{GRUBER-1993-Ontologies}. Descrições de serviços web anotadas com conceitos de uma ontologia são chamadas de serviços web semânticos~\cite{CARDOSO-2006-Semantic-Web-Services}. Diferentes tecnologias e padrões web podem ser utilizados para anotar semanticamente descrições de serviços web, tais como: \textit{Web Ontology Language for Services} (OWL-S)~\cite{W3C-2004-OWL-S}, \textit{Semantic Annotations for WSDL and XML Schema} (SAWSDL)~\cite{W3C-2007-SAWSDL} e \textit{Web Service Modeling Ontology Lite} (WSMO-Lite)~\cite{W3C-2010-WSMO-Lite}. Dentre estes padrões, SAWSDL destaca-se em razão de sua simplicidade.

%%%%%%%%%%%%%%%%%%%%%%%%%%%%%%%%%%%%%%%%%%%%%%%%%%%
% GAP - Suporte ferramental
%%%%%%%%%%%%%%%%%%%%%%%%%%%%%%%%%%%%%%%%%%%%%%%%%%%

O processo de anotação semântica deve ter suporte ferramental adequado. Contudo, a disponibilidade de ferramentas de suporte à anotação semântica é ainda limitada. A baixa disponibilidade de suporte ferramental para a anotação semântica de serviços web dificulta uma maior adoção da tecnologia de serviços web semânticos. Atualmente, exemplos de ferramentas para suporte à anotação semântica utilizando a abordagem SAWSDL incluem: Radiant~\cite{BELHAJJAME-EMBURY-2014-Radiant}, Iridescent~\cite{STAVROPOULOS-2013-Iridescent} e EasyWSDL~\cite{EasySAWSDL-2016}.
%e OWL-S Editor~\cite{SAADATI-DENKER_OWL-S-Editor, SCICLUNA_OWL-S-Editor}.

%O processo de anotação semântica deve ter suporte ferramental adequado. Contudo, a disponibilidade de ferramentas de suporte ainda é limitada. A busca e utilização das ferramentas atualmente disponíveis ainda são considerados grandes desafios. As atuais ferramentas de anotação semântica são disponibilizadas nos formatos \textit{standalone}, \textit{plugins} e/ou bibliotecas de desenvolvimento. A utilização das ferramentas distribuídas no formato \textit{standalone} exigem de seus usuários um intensa busca pelos artefatos computacionais (arquivos executáveis) e suas dependências, como \textit{frameworks} e demais componentes. A utilização das ferramentas distribuídas no formato de \textit{plugins} exigem de seus usuários uma intensa busca por ferramentas (hospedeiras) que suportam tais \textit{plugins}, além de exigir conhecimento das ferramentas hospedeiras em si. Por fim, a utilização das ferramentas distribuídas como bibliotecas de desenvolvimento exigem de seus usuários uma intensa busca das bibliotecas em si, bem como o conhecimento das linguagens de programação com as quais as bibliotecas serão utilizadas.
%Neste sentido, o processo de anotação semântica poderia ser facilitado se as ferramentas fossem disponibilizadas no formato de aplicações web.

%Tais características dificultam a adoção e o desenvolvimento de SWS, já que exigem de seus usuários a busca dos artefatos computacionais para que sejam executados, das ferramentas que suportam os \textit{plugins} ou das linguagens de programação que suportem as bibliotecas de desenvolvimento.

%%%%%%%%%%%%%%%%%%%%%%%%%%%%%%%%%%%%%%%%%%%%%%%%%%%
% GAP - Nível de abstração / Notações visuais
%%%%%%%%%%%%%%%%%%%%%%%%%%%%%%%%%%%%%%%%%%%%%%%%%%%

Notações visuais são uma das formas mais antigas e efetivas de representação do conhecimento~\cite{DAVIES-1990-Egyptian-Hieroglyphs, MATHEWS-Classic-1991-Maya_Emblem_Glyphs, KAMESWARA-2005-Prehistoric-Astronomy-India}. Diferentes tipos de recursos visuais, tais como formas geométricas, cores, ícones, texturas e brilhos, podem ser utilizados para criar modelos em um dado domínio~\cite{SMITH-MORIARTY-KENNEY-BARBATSIS-2004-Handbook-Visual-Communication, MOODY-2009-Physics-Notation}. Adicionalmente, notações visuais são mais eficazes para a comunicação e a transmissão de informações do que outras formas de comunicação, incluindo a comunicação verbal e textual~\cite{MOODY-2009-Physics-Notation}. Tal característica advém da melhor capacidade do cérebro humano em processar (paralelamente) representações visuais (espaciais).

Atualmente, notações visuais tem sido amplamente utilizadas em diversas áreas de conhecimento. Por exemplo, \textit{Unified Modeling Language} (UML)~\cite{OMG-2017-UML} tem sido utilizada na representação de artefatos de \textit{software}; \textit{Business Process Management Notation} (BMPN)~\cite{OMG-2011-BPMN} tem sido utilizada na representação de processos de negócio; \textit{Systems Biology Graphical Notation} (SBGN)~\cite{NOVERE-BUCKA-MI-MOODIE-SCHREIBER-SOROKIN-2009-SBGN, VASUNDRA-LENOVERE-WALTEMATH-WOLKENHAUER-2018-SBGN} tem sido utilizada na representação de modelos biológicos \textit{in silico}; enquanto que \textit{Synthetic Biology Open Language} (SBOL)~\cite{QUINN-COX-ADLER-2015-SBOL} tem sido utilizada na representação de modelos de engenharia genética.

As ferramentas disponíveis de suporte ao desenvolvimento de serviços web semânticos possuem baixo nível de abstração. As abordagens de anotação semântica implementadas pelas atuais ferramentas pressupõem que as anotações devam ser inseridas diretamente nas descrições de serviços WSDL/XML, exigindo de seus usuários um extenso conhecimento técnico. Tanto os elementos que compõem uma descrição de serviço web quanto os conceitos e relacionamentos de uma ontologia poderiam ser mais facilmente compreendidos por meio de notações visuais ao invés das suas representações textuais.

%%%%%%%%%%%%%%%%%%%%%%%%%%%%%%%%%%%%%%%%%%%%%%%%%%%
% GAP - Trabalho colaborativo
%%%%%%%%%%%%%%%%%%%%%%%%%%%%%%%%%%%%%%%%%%%%%%%%%%%

Por fim, nem sempre a tarefa de anotar semanticamente um serviço web é realizada pela mesma pessoa (equipe) que desenvolveu o serviço. Esta tarefa pode ser realizada por um especialista de domínio, i.e., uma pessoa com amplo conhecimento da área de aplicação, mas não necessariamente com conhecimento técnico de serviços web e padrões relacionados. Neste sentido, o processo de anotação semântica deveria, de forma ideal, envolver a participação de diferentes atores, desde desenvolvedores de serviços até especialistas de domínio. Como resultado, o processo de anotação semântica seria facilitado e permitiria que usuários com diferentes especializações colaborassem com sua realização. Porém, atualmente, as ferramentas disponíveis para a anotação semântica não oferecem recursos que deem suporte ao trabalho colaborativo.

%Como resultado, o processo de anotação semântica seria mais eficiente e eficaz, visto que tanto desenvolvedores de serviços quanto especialistas de domínios poderiam trabalhar colaborativamente, simultaneamente e remotamente na criação de serviços web semânticos. 
%O trabalho colaborativo também poderia ser realizado simultaneamente e remotamente.