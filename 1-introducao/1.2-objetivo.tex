\section{Objetivo}\label{1-introducao-objetivo}

Este trabalho tem por objetivo principal investigar o suporte ao desenvolvimento colaborativo de serviços web semânticos por meio de notações gráficas, segundo a abordagem \textit{Semantic Annotations for WSDL and XML Schema} (SAWSDL)~\cite{W3C-2007-SAWSDL}. Neste sentido, este trabalho possui os seguintes objetivos específicos: i) propor uma notação visual para representar descrições WSDL de serviços web e anotações semânticas associadas segundo o padrão SAWSDL; ii) prover suporte à anotação gráfica e colaborativa por meio do desenvolvimento de uma aplicação web.

A utilização de tecnologias web na implementação de uma nova ferramenta facilitará o acesso a um suporte ferramental adequado. O suporte à anotação semântica por meio de notações visuais facilitará a criação de serviços web semânticos, visto que detalhes técnicos serão abstraídos na anotação semântica. Finalmente, o suporte à edição colaborativa permitirá que usuários com diferentes especializações possam anotar semanticamente serviços web de forma colaborativa, simultânea e remota. Com o desenvolvimento de uma ferramenta web de suporte à anotação semântica, de fácil distribuição, com maior usabilidade e com suporte a trabalho colaborativo, acreditamos que o desenvolvimento e a utilização de serviços web semânticos em geral e serviços SAWSDL em particular seja estimulado.

%Com isso, tecnologias relacionadas à serviços web semânticos poderiam ter uma maior adoção pelo mercado.

%A investigação sobre diferentes abordagens para anotação semântica, sobre diferentes abordagens para edição colaborativa, sobre diferentes notações gráficas para representação de serviços web semânticos e ontologias, bem como sobre o atual suporte ferramental para a visualização e a anotação semântica de serviços web contribuem para a construção uma base sólida de conhecimento para o objetivo principal possa ser atingido.

%Adicionalmente, este trabalho tem como objetivo secundário o desenvolvimento de uma ferramenta web que dê suporte ao desenvolvimento de serviços web semânticos. Esta nova ferramenta será mais facilmente distribuída, por ser uma ferramenta web. A fim de suportar uma maior uma melhor usabilidade em comparação às outras ferramentas de suporte à anotação semântica, esta nova ferramenta será baseada em notações gráficas, permitindo abstrair detalhes e conceitos técnicos de seus usuários. Por fim, esta nova ferramenta também contribuirá para uma maior eficiência no desenvolvimento de serviços web, dado que a anotação semântica poderá ser realizada de forma  colaborativa e simultânea entre diferentes usuários em diferentes geolocalizações.