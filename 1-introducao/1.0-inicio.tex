Serviços web semânticos representam uma solução tecnológica para auxiliar na descoberta, no reuso e na composição de serviços web. Um serviço web semântico é criado por meio de anotações semânticas realizadas na descrição de um serviço web. Diferentes padrões de anotação semântica surgiram com o objetivo de prover suporte ao desenvolvimento de serviços web semânticos. O padrão SAWSDL consiste em uma solução simples para o desenvolvimento de serviços web semânticos. Contudo, o suporte ferramental para a anotação semântica é limitado. Atualmente, as ferramentas disponíveis exigem de seus usuários um extenso conhecimento de tecnologias e padrões envolvidos na anotação semântica. Adicionalmente, os formatos de distribuição destas ferramentas dificultam o suporte para a criação e a adoção de serviços web semânticos. Por fim, as atuais ferramentas disponíveis não proveem suporte à anotação semântica de forma colaborativa.

O restante deste capítulo está estruturado da seguinte forma: a seção \ref{1-introducao-motivacao} apresenta a motivação para o desenvolvimento colaborativo de serviços web semânticos por meio de notações visuais; a seção \ref{1-introducao-objetivo} apresenta os objetivos do trabalho; a seção \ref{1-introducao-metodologia} apresenta a metodologia utilizada para desenvolvimento deste trabalho; e, por fim, a seção \ref{1-introducao-estrutura-documento} apresenta a estrutura dos demais capítulos desta dissertação.