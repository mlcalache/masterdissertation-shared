\section{Metodologia}\label{1-introducao-metodologia}

Inicialmente, realizamos uma revisão bibliográfica sobre serviços web, ontologias, abordagens de anotações semânticas para serviços web, notações visuais e sistemas de edição colaborativa. Adicionalmente, realizamos uma pesquisa sobre as principais ferramentas de anotação semântica propostas na literatura.

Após completar os estudos de fundamentação teórica do trabalho, realizamos uma pesquisa de diferentes \textit{frameworks} e bibliotecas para o suporte ao desenvolvimento da nossa ferramenta de anotação colaborativa. Adicionalmente, pesquisamos diferentes abordagens tecnológicas para permitir a anotação semântica de forma colaborativa. Estas atividades tiveram por objetivo construir uma base sólida de conhecimento para compreender os desafios das diferentes abordagens para anotação semântica de um serviço web, bem como para planejar e projetar a criação da nova ferramenta.

Após os estudos que embasaram o desenvolvimento deste trabalho, propusemos uma notação visual para representar os diferentes tipos de elementos que compõem uma descrição de serviço web WSDL 2.0, bem como uma notação visual para representar hierarquicamente conceitos de uma ontologia e os diferentes tipos de anotações semânticas definidos em SAWSDL.

%Adicionalmente, a nova ferramenta tinha como requisito a anotação semântica por meio de notações gráficas que possibilitassem abstrair os dados sintáticos envolvidos na anotação semântica. Ou seja, uma interface gráfica que fosse independente de conhecimento técnico envolvido na descrição de um serviço web e na anotação semântica em si.

Em seguida, levantamos os requisitos funcionais e não-funcionais da ferramenta e criamos modelos da interface de usuário (UI - \textit{User Interface}), também chamados de \textit{mockups}, para suas principais funcionalidades. A fase de análise dos requisitos também possibilitou a total compreensão do escopo de desenvolvimento da nova ferramenta, bem como a criação da estrutura analítica de projeto (EAP)~\cite{PMI-2017-PMBOK}. A EAP auxiliou o desenvolvimento do projeto pois forneceu uma decomposição hierárquica orientada à entrega do trabalho a ser executado durante o projeto.

%Por meio da EAP, foi possível quebrarmos os requisitos em pacotes de trabalho menores. Com os pacotes de trabalho menores, foi possível priorizarmos os requisitos, controlarmos as entregas e atendermos aos prazos de desenvolvimento e validação da ferramenta em relação ao prazo de entrega e defesa deste trabalho.

%bem como a priorização dos requisitos. Por meio da priorização dos requisitos, foi possível criar um cronograma para as atividades de desenvolvimento, bem como a quebra dos requisitos em pacotes de trabalho menores, tendo como resultado uma estrutura analítica de projetos (EAP).

Após a fase de análise dos requisitos, iniciamos o desenvolvimento da nova ferramenta de suporte à anotação semântica. A ferramenta foi desenvolvida utilizando a linguagem de programação C\#, do \textit{Microsoft .NET Framework}, bem como um conjunto de padrões e práticas de desenvolvimento que facilitam a estruturação e o reuso do código, tais como padrões de projeto (\textit{design-patterns}); os princípios \textit{Single Responsibility}, \textit{Open/Closed}, \textit{Liskov Substitution}, \textit{Interface Segregation} e \textit{Dependency Inversion} (SOLID); a abordagem de desenvolvimento \textit{Domain-Driven Design} (DDD)~\cite{EVANS-2004-DDD}; entre outros.

Finalmente, de modo a validar o funcionamento da nova ferramenta, desenvolvemos dois estudos de caso. O primeiro estudo de caso envolveu a anotação semântica de um serviço no domínio literário. Este estudo de caso foi utilizado para demonstrar o uso da ferramenta de forma aplicada e prática. Já o segundo estudo de caso envolveu a anotação semântica de um serviço no domínio cinematográfico. Este estudo de caso foi então utilizado para realizar uma avaliação empírica da ferramenta com um grupo de usuários. Os dois estudos de caso possibilitaram avaliarmos as funcionalidades e a usabilidade da ferramenta desenvolvida, bem como ressaltarmos os principais diferenciais (benefícios) em relação às ferramentas já existentes e propostas na literatura.

%descrições de serviços web para apresentarmos um estudo de caso e para avaliarmos a ferramenta. Optamos por desenvolver uma descrição de serviço web no domínio de livros para o estudo de caso. Desenvolvemos também outra descrição de serviço web no domínio de filmes para a validação da ferramenta. Dados os dois domínios, pesquisamos ontologias que pudessem ser utilizadas para as descrições de serviços que desenvolvemos. Um manual de uso da ferramenta também foi elaborado. Adicionalmente, vídeos de como utilizar a ferramenta foram criados a fim de contribuir, juntamente com o manual, para um melhor entendimento do uso dos recursos providos pela ferramenta. Com isso, criamos um estudo de caso detalhado da utilização da ferramenta e validamos a ferramenta com usuários utilizando descrições de serviços e ontologias diferentes para cada objetivo.

%Para a validação da ferramenta, a descrição de serviço web desenvolvida no domínio de filmes e as ontologias encontradas para o mesmo domínio foram compartilhadas entre um conjunto de usuários. Adicionalmente, o manual e os vídeos instrucionais também foram compartilhados entre os mesmos usuários. Por fim, elaboramos um questionário para avaliarmos o uso da ferramenta entre os usuários. Portanto, por meio da descrição de serviço, das ontologias, do manual, dos vídeos instrucionais e do questionário, realizamos uma avaliação empírica.

%Como resultado, a avaliação possibilitou que apresentássemos a ferramenta para a comunidade científica, contribuindo para um melhor suporte ferramental para anotação semântica de serviços web.