Este trabalho teve por objetivo a investigação do suporte ferramental para o desenvolvimento de serviços web semânticos, segundo a abordagem SAWSDL`\cite{W3C-2007-SAWSDL}, por meio de notações visuais e de forma colaborativa. Este capítulo apresenta as principais contribuições deste trabalho, além de discutir sobre o uso e adoção de serviços web semânticos pela comunidade e apresentar os trabalhos futuros provenientes deste trabalho.

O capítulo está estruturado da seguinte forma: a seção \ref{7-conclusao-principais-contribuicoes} apresenta as principais contribuições deste trabalho; a seção \ref{7-conclusao-discussao} discute a abordagem proposta para a anotação semântica de serviços web; e, por fim, a seção \ref{7-conclusao-trabalhos-futuros} apresenta os trabalhos futuros.